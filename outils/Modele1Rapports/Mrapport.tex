%\documentclass[french, a4, 12pt, titlepage]{article}
\documentclass[french, a4, 10pt]{article} % si la table des matiï¿œes est petite
\usepackage[utf8]{inputenc}
%\usepackage[latin1]{inputenc}
\usepackage[francais]{babel}
% Modification des marges ------------------------------
\oddsidemargin -4mm 	% Marge de gauche -4mm
\textwidth 17cm 	% Largeur de gauche = 17cm
\textheight 22cm 	% Hauteur du texte = 22cm
\parindent 0cm		% Pas d'indentation de paragraphe
% -----------------------------------------------------
\usepackage[T1]{fontenc}
\usepackage{graphicx,color, caption2}
\usepackage{epsfig}
\usepackage{fancyhdr}
%\usepackage{fancyvrb}
\usepackage{textcomp}
\pagestyle{fancy}
\usepackage{listings}		% pour incorporer des sources
\usepackage[francais]{layout}	% pour obtenir le layout
\usepackage{fullpage}		% pour obtenir le layout
\usepackage{makeidx}		% pour créer une table d'index
%\usepackage{shadow}		% pour faire des encadrements
\begin{document}
% Titres sur  chaque page -----------------------------
\lhead{ } % Haut gauche
%\chead{Haut centre}
\rhead{Ecrire un rapport en latex}
\lfoot{\copyright{Jaumain Jean-Claude}}
\cfoot{ } % Bas centre - obligatoire !
\rfoot{\thepage} % Bas droite
\renewcommand{\headrulewidth}{0.4pt} 
\renewcommand{\footrulewidth}{0.4pt} 
% Numérotation et table des matières -----------------
\setcounter{tocdepth}{1}    % fixe la profondeur de la table des matières
\setcounter{secnumdepth}{5} % fixe la profondeur de la numérotation des sections et paragraph
% Page de garde --------------------------------------
\title{\emph{\textbf{Écrire un rapport en Latex}}}
\author{Jaumain J-C}
\date{15 septembre 2010}
\maketitle
\tableofcontents
% Inclusion des textes
30 mtime=1564061967.224818705
30 atime=1585422196.382681604
30 ctime=1564061967.224818705

\section{Mathématiques}
% use package amsmath, amsfont, amssymb peuvent être utiles.
\index{équation}
Quelques exemples des possibilités mathématiques :

La fonction $e^x$ est strictement croissante sur $R$ et $\forall x \in R$.

\begin{equation}
\frac{\partial}{\partial y}\int_E f(x,y)\,dx = \int_E \frac{\partial f(x,y)}{\partial y}\,dx
\end{equation}

$$\lim_{x \to +\infty} \frac{\ln x}{x} = 0$$

$$ 10 \textrm{ dixièmes} = 1 $$

\begin{equation}
\sum_{k=1}^n k = \frac{n(n+1)}{2}
\end{equation}

\begin{equation}
\int_0^{+\infty} x^n e^{-x}\,dx = n!
\end{equation}

$$\left \{\begin{array}{c}x+y=1\\x-y=1\end{array}\right.$$

\begin{equation}
\left( \begin{array}{cc} a & b \\ c & d \end{array} \right) \cdot
\left( \begin{array}{cc} 0 & 1 \\ 0 & 0 \end{array} \right) =
\left( \begin{array}{cc} 0 & a \\ 0 & c \end{array} \right)
\end{equation}

$$\widehat{ab} + \widehat{bc} + \widehat{cb} = 180$$

$$\overrightarrow{ab} + \overrightarrow{ac} = \overrightarrow{ad}$$
 % pas de saut de page
\section{Intégrer une source}
\subsection{Configuration de lstlisting}
% Le manuel se trouve dans le howto ou dans CTAN sous le nom de listings.pdf
La commande lstset permet de fixer la présentation des sources. Il n'est pas conseillé d'utiliser utf8 pour les sources si le document n'est pas en utf8.
\lstset{language={},%C,Assembleur, TeX, tcl, basic, cobol, fortran, logo, make, pascal, perl, prolog, {}
	literate={â}{{\^a}}1 {ê}{{\^e}}1 {î}{{\^i}}1 {ô}{{\^o}}1 {û}{{\^u}}1
		 {ä}{{\"a}}1 {ë}{{\"e}}1 {ï}{{\"i}}1 {ö}{{\"o}}1 {ü}{{\"u}}1
		 {à}{{\`a}}1 {é}{{\'e}}1 {è}{{\`e}}1 {ù}{{\`u}}1 
		 {Â}{{\^A}}1 {Ê}{{\^E}}1 {Î}{{\^I}}1 {Ô}{{\^O}}1 {Û}{{\^U}}1
		 {Ä}{{\"A}}1 {Ë}{{\"E}}1 {Ï}{{\"I}}1 {Ö}{{\"O}}1 {Ü}{{\"U}}1
		 {À}{{\`A}}1 {É}{{\'E}}1 {È}{{\`E}}1 {Ù}{{\`U}}1,
	commentstyle=\scriptsize\ttfamily\slshape, % style des commentaires
	basicstyle=\scriptsize\ttfamily, % style par défaut
	keywordstyle=\scriptsize\rmfamily\bfseries,% style des mots-clés
	backgroundcolor=\color[rgb]{.95,.95,.95}, % couleur de fond : gris clair
	framerule=0.5pt,% Taille des bords
	frame=trbl,% Style du cadre
	frameround=tttt, % Bords arrondis 
	tabsize=3, % Taille des tabulations
%	extendedchars=\true, % Incompatible avec utf8 et literate
	inputencoding=utf8,
	showspaces=false, % Ne montre pas les espaces 
	showstringspaces=false, % Ne montre pas les espaces entre ''
	xrightmargin=-1cm, % Retrait gauche 
	xleftmargin=-1cm, % Retrait droit
	escapechar=@}  % Caractère d'échappement, permet des commandes latex dans la source
% -----------------------------------------------------
\subsection{Intégrer une source dans le texte}
\begin{lstlisting}
/*---------------------------------------------------------------------------------------
NOM      : Exemple.c
CLASSE   : Applications - Latex - Illustration
OBJET    : Sert d'exemple pour inclure une source en latex
         : Dans ce ces, ce programme affiche Hello
HOWTO    : gcc Exemple.c -o Exemple; ./Exemple
AUTEUR   : J.C. Jaumain, le 3/11/2010
BUGS     :  /
REMARQUE : Impose lstset {escapechar=@\symbol{64}@} pour l'interprétation des balises latex
----------------------------------------------------------------------------------------*/
main() {
	int i;  // Pour récupérer le nombre de caractères écrits
	tab[10] Buffer='Hello'; // Le buffer
	i=write(1,Buffer,5); // La @$\frac{1}{2}$@ du buffer
	exit(0);
}
\end{lstlisting}

L'intérêt de cette technique est de figer le source et d'avoir un document autonome

\subsection{Intégrer une source d'un fichier}

\lstinputlisting{Exemple.c}

L'intérêt de cette technique est d'avoir un source toujours "à jour".




30 mtime=1564061966.944818697
28 atime=1585422196.5546816
30 ctime=1564061966.944818697

\section{Mise en page}

\subsection{Marges...}

Une série de variables définissent la mise en page. En utilisant les packages [francais]{layout} et {fullpage}, on peut utiliser la commande layout qui permet d'ajouter une page qui dessine la présentation d'une page et les noms des variables assignées.

\layout

On peut ensuite modifier ce que l'on souhaite :
\begin{verbatim}
% Modification des marges ------------------------------
\oddsidemargin -4mm 	% Marge de gauche -4mm
\textwidth 17cm 	% Largeur de gauche = 17cm
\textheight 22cm 	% Hauteur du texte = 22cm
\parindent 0cm		% Pas d'indentation de paragraphe
% -----------------------------------------------------
\end{verbatim}

\subsection{Niveaux}

Vous avez droit à la structure

\begin{list}{$\bullet$}{}
\item part avec saut de page
\item chapter : niveau 0
\item section : niveau 1
\item subsection : niveau 2
\item subsubsection : niveau 3
\item paragraph : niveau 4
\item subparagraph : niveau 5
\end{list}

La variable secnumdepth permet de limiter la numérotation des niveaux. Par exemple, la valeur 5 permet d'avoir une numérotation du style 1.2.3.2.1.2 pour le subparagraph\footnote{De la même façon, on peut limiter le nombre de niveaux affichés dans une table des matières avec la variable tocdepth }. 
(Il est toujours possible d'insérer une note de bas de page avec \verb+\footnote+)

\subsection{Cadres}

Il est possible d'encadrer un mot avec \fbox{box}

%\shabox{On peut aussi encadrer tout un paragraphe avec l'environnement shabox à condition de déclarer le package shadow avant le début du document}.

30 mtime=1564061967.180818703
30 atime=1585422196.566681599
30 ctime=1564061967.180818703

\section{Programmation}

% \subsection{Variables}   REPRENDRE LES SLIDES
% 
% Deux sortes de variables : les entiers et les dimensions. Les dimensions sont toujours accompagnées d'une unité. (mm, cm, pt,...)
% 
% \subsubsection{variables}
% \begin{itemize}
% \item Création : 
% \item Assignation :
% \item Incrémentation :
% \item Tests :
% \end{itemize}
% 
% \subsubsection{variables prédéfinies}
% \begin{itemize}
% \item 
% \item 
% \item 
% \item 
% \item 
% \item 
% \item 
% \item 
% \item 
% \item 
% \item 
% \end{itemize}
% 
% \subsubsection{dimensions}
% \begin{itemize}
% \item Création : 
% \item Assignation :
% \item Incrémentation :
% \item Tests :
% \end{itemize}
% 
% \subsubsection{dimensions prédéfinies}
% \begin{itemize}
% \item 
% \item 
% \item 
% \item 
% \item 
% \item 
% \item 
% \item 
% \item 
% \item 
% \item 
% \end{itemize}
% 
% 
% \section{contrôles}
% 
\section{commandes}

\begin{verbatim}
\newcommand{\NomCmd}[argc][def1]
{ Commandes où les arguments s'appellent #1 #2...
}
ou \renewcommand...
\end{verbatim}
où
\begin{itemize}
\item NomCmd est le nom donné à la commande 
\item argc est le nombre d'argument, 0 par défaut
\item def1 est une valeur par défaut pour le premier argument.
\item newcommand pour définir une nouvelle commande dont le nom n'existe pas 
\item renewcommand pour redéfinir une commande dont le nom existe déjà 
\end{itemize}

\subsection{Exemples:}

\newcommand{\EtatCivil}[3]{\textbf{#1} #2 #3}
\EtatCivil{M.}{Albert}{Einstein}\\
\renewcommand{\EtatCivil}[3][M.]{\textbf{#1} #2 #3}
\EtatCivil{Albert}{Einstein}\\
\EtatCivil[Mme]{Julio}{Curie}\\

% \section{environnement}    NE FONCTIONNE PAS !!!!
% 
% \begin{verbatim}
% \newenvironment{NomEnv}[argc][def1]
% {sequenceDebut}{SectionFin} où les arguments s'appellent #1 #2...
% \end{verbatim}
% où
% \begin{itemize}
% \item NomEnv est le nom donné à l'environnement 
% \item argc est le nombre d'argument, 0 par défaut
% \item def1, optionnel, est une valeur par défaut pour le premier argument.
% \item newenvironment pour définir un nouvel environnement dont le nom n'existe pas 
% \end{itemize}
% 
% \subsection{Exemples:}
% 
% \newenvironment{MonTitre}[1]{\
% begin{center}\textbf{#1}\\}{\end{center}}
% 
% \begin{MonTitre}{Première ligne}
% Ligne deuxième
% \end{MonTitre}



%\include{Latex}
%\include{Latex}
\printindex			% Impression de la table des index
\end{document}
